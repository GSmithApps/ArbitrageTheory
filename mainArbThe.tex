\documentclass{article}
\usepackage[utf8]{inputenc}

\title{Arbitrage Theory}
\author{Grant Smith}
\date{Spring 2022}

\begin{document}

\maketitle

\section{Introduction}

The whole point of arbitrage theory is that if arbitrage exists, there is something seriously wrong with the pricing in the market. Something so incorrect that we cannot even generate a pretend distribution, no matter how unrealistic, that would make it look like everyone was behaving rationally. 

But if arbitrage doesn't exist, then despite whatever the reality of the probability of the universe, we can at least make up a probability distribution and pretend that that's the real distribution and say everyone is acting rationally according to that distribution. And this is really useful for us.  So we really want the absence of arbitrage. And if it isn't there, then we have to do something else. 

But make no mistake - if arbitrage doesn't exist, then the probability distribution we make up doesn't have to match reality. And in fact if it doesn't, then you can still make a profit. 

Arbitrage says that at least you're going to have some negative no matter what you hold. Having some negative is fine as a normal investor, but at least we can't argue that there's no way you possibly wouldn't do it.

\end{document}
