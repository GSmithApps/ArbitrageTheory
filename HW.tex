\documentclass[]{book}

%These tell TeX which packages to use.
\usepackage{array,epsfig}
\usepackage{amsmath}
\usepackage{amsfonts}
\usepackage{amssymb}
\usepackage{amsxtra}
\usepackage{amsthm}
\usepackage{mathrsfs}
\usepackage{color}

%Here I define some theorem styles and shortcut commands for symbols I use often
\theoremstyle{definition}
\newtheorem{defn}{Definition}
\newtheorem{thm}{Theorem}
\newtheorem{cor}{Corollary}
\newtheorem*{rmk}{Remark}
\newtheorem{lem}{Lemma}
\newtheorem*{joke}{Joke}
\newtheorem{ex}{Example}
\newtheorem*{soln}{Solution}
\newtheorem{prop}{Proposition}

\newcommand{\lra}{\longrightarrow}
\newcommand{\ra}{\rightarrow}
\newcommand{\surj}{\twoheadrightarrow}
\newcommand{\graph}{\mathrm{graph}}
\newcommand{\bb}[1]{\mathbb{#1}}
\newcommand{\Z}{\bb{Z}}
\newcommand{\Q}{\bb{Q}}
\newcommand{\R}{\bb{R}}
\newcommand{\C}{\bb{C}}
\newcommand{\N}{\bb{N}}
\newcommand{\M}{\mathbf{M}}
\newcommand{\m}{\mathbf{m}}
\newcommand{\MM}{\mathscr{M}}
\newcommand{\HH}{\mathscr{H}}
\newcommand{\Om}{\Omega}
\newcommand{\Ho}{\in\HH(\Om)}
\newcommand{\bd}{\partial}
\newcommand{\del}{\partial}
\newcommand{\bardel}{\overline\partial}
\newcommand{\textdf}[1]{\textbf{\textsf{#1}}\index{#1}}
\newcommand{\img}{\mathrm{img}}
\newcommand{\ip}[2]{\left\langle{#1},{#2}\right\rangle}
\newcommand{\inter}[1]{\mathrm{int}{#1}}
\newcommand{\exter}[1]{\mathrm{ext}{#1}}
\newcommand{\cl}[1]{\mathrm{cl}{#1}}
\newcommand{\ds}{\displaystyle}
\newcommand{\vol}{\mathrm{vol}}
\newcommand{\cnt}{\mathrm{ct}}
\newcommand{\osc}{\mathrm{osc}}
\newcommand{\LL}{\mathbf{L}}
\newcommand{\UU}{\mathbf{U}}
\newcommand{\support}{\mathrm{support}}
\newcommand{\AND}{\;\wedge\;}
\newcommand{\OR}{\;\vee\;}
\newcommand{\Oset}{\varnothing}
\newcommand{\st}{\ni}
\newcommand{\wh}{\widehat}

%Pagination stuff.
\setlength{\topmargin}{-.3 in}
\setlength{\oddsidemargin}{0in}
\setlength{\evensidemargin}{0in}
\setlength{\textheight}{9.in}
\setlength{\textwidth}{6.5in}
\pagestyle{empty}



\begin{document}


\begin{center}
{\Large Econ 780 \hspace{0.5cm} HW Week 2}\\
\textbf{Minh Cao, Grant Smith, Ella Barnes, Alexander Erwin and Mark Coomes}\\ %You should put your name here
Due:  %You should write the date here.
\end{center}

\vspace{0.2 cm}


\subsection*{Exercises for chapter 2}

\begin{enumerate}
\item Question 1:  Familiarize yourself with InkScape and LaTeX. Done.
\item
Question 2.a


\begin{proof}
    First, we know that ${B_1}'={B_1}''$. Using this, we can say $R'=R''$ by the following:
    $$1+R'={B_1}'={B_1}''=1+R'' \implies R'=R''.$$
    Let $Q'$ be the martingale measure for model prime, by definition 2.4. ${S_0}'= \frac{1}{1+R'}E^{Q'}[S_1]$, multiply both sides by $\frac{1}{\alpha}$ and using the linearity of expectation operation, we have:\\
    $$\frac{1}{\alpha}{S_0}'= \frac{1}{1+R'}E^{Q'}[\frac{1}{\alpha}{S_1}']$$
    We also know, from the problem statement, that ${S_t}'=\alpha {S_t}''$ or, rearranged, $\frac{1}{\alpha} {S_t}'={S_t}''$. Substituting this and $R''$ for $R'$, we have:
    $${S_0}''= \frac{1}{1+R''}E^{Q'}[{S_1}'']$$
    And from the definition of a martingale measure, we can say stock 2's martingale measure equation is
    $${S_0}''= \frac{1}{1+R''}E^{Q''}[{S_1}'']$$
    which implies that $Q' = Q''$ 
\end{proof}

2.b
\begin{proof}
As above, we know that $R'=R''$, which we will simply call $R$. Furthermore, using the definition of martingale measures, we know the following:

$${S_0}'= \frac{1}{1+R'}E^{Q'}[{S_1}']=\frac{1}{1+R}E^{Q'}[{S_1}']$$
$${S_0}''= \frac{1}{1+R''}E^{Q''}[{S_1}'']= \frac{1}{1+R}E^{Q''}[{S_1}'']$$

Also from the problem statement, we know that ${S_1}' = \alpha{S_1}'' $. Doing manipulations similar to above, we multiply the second equation by $\alpha$ and use the linearity of expectation to get 

$$\alpha{S_0}''= \frac{1}{1+R}E^{Q''}[\alpha{S_1}'']= \frac{1}{1+R}E^{Q''}[{S_1}']$$

Furthermore, using ${S_0}' > \alpha{S_0}''$, we get the following:

$$\frac{1}{1+R}E^{Q'}[{S_1}'] > \frac{1}{1+R}E^{Q''}[{S_1}']$$
$$E^{Q'}[{S_1}'] > E^{Q''}[{S_1}']$$

Which implies that $Q'$ puts more weight on the good outcome than $Q''$

\end{proof}

2.c. 
\begin{proof}

    First, we know that ${B_1}'<{B_1}''$, so  $1+R'<1+R''$, which implies:
    
    $$\frac{1}{1+R'}> \frac{1}{1+R''}$$
    
    Or
    
    $$ \frac{1}{1+R''} < \frac{1}{1+R'}$$

Using the definition of martingale measures, we know the following:

$${S_0}'= \frac{1}{1+R'}E^{Q'}[{S_1}']$$
$${S_0}''= \frac{1}{1+R''}E^{Q''}[{S_1}'']$$

Doing manipulations similar to above, we multiply the second equation by $\alpha$ and use the linearity of expectation to get 

$$\alpha{S_0}''= \frac{1}{1+R''}E^{Q''}[\alpha{S_1}'']$$

Also from the problem statement, we know that ${S_t}' = \alpha{S_t}'' $. Substituting gives:

$${S_0}'= \frac{1}{1+R''}E^{Q''}[{S_1}']$$

And using equation the first equation above:

$$\frac{1}{1+R'}E^{Q'}[{S_1}']= \frac{1}{1+R''}E^{Q''}[{S_1}']$$

And using the bond rate inequality, 

$$\frac{1}{1+R'}E^{Q'}[{S_1}']= \frac{1}{1+R''}E^{Q''}[{S_1}'] < \frac{1}{1+R'}E^{Q''}[{S_1}']  $$

And stated clearly:

$$E^{Q'}[{S_1}'] < E^{Q''}[{S_1}']  $$

Hence the double prime martingale measure puts more weight in the good outcome compared to the prime martingale measure.\\

\end{proof}

\item Question 2.1.
\begin{proof}


From definition 2.4, we have: $S_0=s = \frac{1}{1+R}E^{Q}[S_1] = \frac{1}{1+R}\left[q_usu+q_dsd \right] =  \frac{1}{1+R}\left[q_usu+ (1-q_u)sd\right] = \frac{1}{1+R}\left[q_usu+ (1-q_u)sd\right] \implies 1+R = q_u(u-d) + d\implies q_u = \frac{(1+R)-d}{u-d}$.\\
Symmetric argument, $q_d = \frac{(u-1+R)}{u-d}$

\end{proof}

\begin{proof}
WLOG, suppose $X_1<X = \phi(Z)$ We can get "free money" by buying the contingentcy plan at price $X_1<\phi(Z)$ and recieve $\phi(Z)>X_1$
\end{proof}

\item Question 4.

In this question, the only thing master is the rate of return of bond, in the original model, let the rate of return bond in the new model be $\beta$.

We have $b_1=B_1 = \beta B_0 = \beta b_0 \implies  \beta = \frac{b_1}{b_0}$

Hence, statement 2.3 become: $d<\frac{b_1}{b_0}<u$.

Statement 2.4 become: $S_0= \frac{b_0}{b_1}E^{Q}[S_1]$.

The formular for Q will become:\\ 
\begin{equation*}
 Q = \left\{
        \begin{array}{ll}
            q_u = \frac{(\frac{b_1}{b_0})-d}{u-d}\\
            q_d = \frac{u-\frac{b_1}{b_0}}{u-d}
        \end{array}
    \right.
\end{equation*}
\item Question 5\\
Yes, by theorem 2.3. we have $d=\frac{2}{3} \leq 1+R =1+0=1 \leq \frac{3}{2}$. Hence the market is arbitrage free.

To create a replica porfolio. Note that this model now have 3 two period.
we have:
$$\phi(\frac{9}{4}) = 1, \phi(1) = 0 , \phi(\frac{4}{9})= 0$$


$$\frac{3}{2} = q_u \frac{9}{4} +(1-q_u)*1 \implies q_u = 0.4$$
Implies in the node facing (1,0) $$X_1 = \frac{1}{1+0}0.4 * \frac{9}{4} + 0.6*1 = 0.4$$
Similarly, in the lower node. $X_1 = 0$.\\
In the original node. $X_0 = \frac{4}{25}$.\\

Now compute h at node 0.
$$x_0 = \frac{\frac{2}{3}0.4-\frac{3}{2}*0}{\frac{3}{2}-\frac{2}{3}} = -\frac{8}{25}$$
Similarly $y_0 = \frac{12}{25}$
Now we are facing claim: $\frac{9}{4}, 0$. using similar strategy, we can compute: $x_1 = -\frac{4}{5},y_1= \frac{2}{3} \frac{1}{\frac{5}{4}} = \frac{8}{15}$.
At node facing the claim $(0,0)$ is $(x_1,y_1) = (0,0)$











\item Question 6

If at node 120, the option price is 50. We can make "free profit".
Let the portfolio $h =(0,0,0)$ at node 80 and $h=(0,0,0)$ at node 40. let $h=( -42.5 -2.5,\frac{95}{120} , 1)$. 
$$h = -42.5-2.5+\frac{95}{120}*120 - 50 = 0$$.\\
Hence the cost of the portfolio h at 120 is 0, at date T the portfolio worth 2.5 dollar. 
\item Question 7.

Part A) Prove by induction that for all $n>1$:

$$n^3 = \frac{n^2(n+1)^2}{4}$$

Base Case:

$$1^3=1=\frac{1^2*2^2}{4}$$

Assume the equation holds at $n$, we need to prove that it holds at $n+1$. So assume the following:

$$1^3 + 2^3 + ... + n^3 = \frac{n^2(n+1)^2}{4}$$

and try to prove:

$$1^3 + 2^3 + ... + n^3 + (n+1)^3 = \frac{(n+1)^2(n+1+1)^2}{4}$$

First, use the assumption and plug into the left side of the equation. Thus:

$$\frac{n^2(n+1)^2}{4} + (n+1)^3 = \frac{(n+1)^2(n+1+1)^2}{4}$$

Now simplify both sides:

$$\frac{x^4+6x^3+13x^2+12x+4}{4}=\frac{x^4+6x^3+13x^2+12x+4}{4}$$

Thus, we have proven that for all $n>1$:

$$n^3 = \frac{n^2(n+1)^2}{4}$$


Part B) Prove by induction that for all $n>1, 4^n + 15n -1$ is divisble by 9. Start with a base case of n = 1

$$4^1 + 15 * 1 - 1 = 4 + 15 -1 = 18 = 9 * 2 + 0$$

Thus, the base case is satisfied. Now we assume that the equation holds at n, and we want to prove that it holds at n+1. Start with the assumption that the equation holds at n:

$$\exists k\in \mathbb{N} s.t.  4^n + 15n - 1 = 9k $$

Now we try to prove the inductive case:

$$\exists k'\in \mathbb{N} s.t.  4^{n+1} + 15(n+1) - 1 = 9k' $$

And we simplify:

$$4^{n+1} + 15(n+1) - 1 =  4^n*4 + 15n+15 - 1 =  4^n*3+15 + 4^n + 15n - 1$$

$$=4^n*3+15 +9k $$

We can drop the 9k because it is certainly divisible by 9.  So we now are left with proving that $4^n*3+15$ is divisible by 9, which we can do by induction. To do so, we show that the base case of $4^1*3 + 15 = 27$ is divisible by 9.  We then show the inductive step that if $4^n * 3 + 15$ is divisible by 9, so must be $4^{n+1} * 3 + 15$.

$$4^{n+1} * 3 + 15 = 4^n * 4 * 3 + 15 = 4^n * 3 * 3 + 4^n * 1 * 3 + 15 = 4^n * 3 * 3 + 9k'$$

In which case we can drop the $9k'$, and we are left with $9 * 4^n$ which is certainly divisible by 9. Thus, for all $n>1$:

$$n^3 = \frac{n^2(n+1)^2}{4}.$$




\item Question 8\\

\begin{proof}



Given:\\
$V_t^h=xB_t+yS_t$ as A\\
$B_t=1+R$ as B\\
$S_t=su^kd^{t-k}$ as C\\
$X_t=\frac{1}{1+R}\frac{uV_t(k)-dV_t(k+1)}{u-d}$ as D\\
$y_t(k)=\frac{1}{S_{t-1}}\frac{V_t(k+1)-V_t(k)}{u-d}$ as E\\
\\
Plugging B, D, and E into A yields:\\
$$V_t^h=\frac{v_t(k)[S_{t-1}u-s_t+v_t{k+1}[-S_d+s_t]}{S_{t-1}(u-d)}$$\\
\\
By simplifying and combining like fractions we find:\\
$$V_t^h=\frac{V_t(k)[S_{t-1}u-S_t]+V_t(k+1)[-S_{t-1}d+S_t]}{S_{t-1}(u-d)}$$\\
\\
We are then able to apply $t=0$, $k=0$\\
$$V_0(0)=\frac{V_0(0)[S_{-1}u-S_0]+V_0(1)[-S_{-1}d+S_0]}{S_{-1}d+S_0}$$\\
\\
Using C, we find that $S_{-1}=\frac{s}{d}$ and $S_0=S$, which we plug in\\
$$V_0(0)=\frac{V_0(0)[\frac{s}{d}u-s+V_0(1)[\frac{-s}{d}d+S]}{\frac{s}{d}(u-d)}=\frac{d}{s}\frac{V_0(0)[\frac{s}{d}u-s]}{(u-d)}$$\\
\\
$$V_0(0)=\frac{dV_0(0)[\frac{u}{d}-1]}{u-d}=\frac{V_0(0)[u-d]}{u-d}$$\\
\\
Which produces the equality:\\
$$V_0(0)=V_0(0)$$\\
\\
Thus validating this equation's preservation of the identity of $V_0(0)$\\



Alternative proof.

for $T=1$.
we have the scheme, $$V_1(1) = \phi(su), V_1(0) = \phi(sk), V_0(0) = \frac{1}{1+R}\left[q_uV_1(1)+q_d(V_1(0)\right]$$

$$q_u = \frac{(1+R)-d}{u-d}, q_d = \frac{(u- 1+R)}{u-d}$$.

Implies $$x_0(0)= \frac{1}{1+R}\frac{uV_1(0)-dV_1(1)}{u-d}$$

$$y_0(0)= \frac{1}{S_0}\frac{V_1(1)-V_1(0)}{u-d}$$

So 2.24 true for $T=1$

Suppose the scheme true for $T=\tau$
we need to show the scheme true for $T=\tau+1$\\

We will Following the same strategy.

given at period $\tau +1$, the value of the contingent claim at node $(t,k)$  is given by $\phi(su^kd^{\tau+1 - k})$. 
if we consider node $(t,k)$ such that $t+k = \tau$, this is binomial model at node $(t,k)$. we can following the scheme to find a replica portfolio for a binomial model.
$$V_{\tau}(k) = \frac{1}{1+R}\left[q_u\phi(su^{k+1} d^{\tau+1-(k+1)})+q_d\phi(su^{k}d^{\tau +1 -k}\right]$$
$$V_{\tau+1}(k) = \phi(su^kd^{\tau+1-k})$$

Substitute the second equation to the first equation.

$$V_{\tau}(k) = \frac{1}{1+R}\left[q_uV_{t+1}(k+1)+q_dV_{\tau +1}(k)\right]$$

and compute the replica portfolio as in the canonical binomial model, we have:

$$x_{\tau}(k) = \frac{1}{1+R} \frac{uV_{\tau}(k) -dV_t(k+1)}{u-d}$$

$$y_{\tau}(k) = \frac{1}{S_{\tau-1}}\frac{V_{\tau}(k+1)-V_{\tau}(k)}{u-d}$$.\\

For a given $0<t<\tau +1$, 
since we assume that the scheme is true for $T=\tau$, at node $(t,k)$, period t. we have:

$$V_{t}(k) = \frac{1}{1+R}\left[q_uV_{t+1}(k+1)+q_dV_{t +1}(k)\right]$$
$$V_{\tau}(k) = \phi(su^kd^{\tau-k})$$.

Incomplete.







\end{proof}









































\end{enumerate}
\end{document}


