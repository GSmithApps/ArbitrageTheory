\documentclass[12pt]{article}

\usepackage[utf8]{inputenc}


\usepackage{amsmath, amssymb, amsfonts, amsthm, mathrsfs, bm}
%\usepackage{array, multirow}
\usepackage{color, float, graphicx, caption, subcaption}
\usepackage{comment}

\usepackage{mathtools}%for paired delimiters

\usepackage{hyperref}

\renewcommand{\arraystretch}{1.0}

\captionsetup{font=small}
\usepackage{setspace}
\setstretch{1.0}
%\setlength\topmargin{0in}
%\setlength\textheight{8in}
%\setlength\textwidth{5.5in}
%\setlength\evensidemargin{.5in}
%\setlength\oddsidemargin{.5in}

\setlength\topmargin{-.45in}
\setlength\textheight{8.5in}
\setlength\textwidth{6in}
\setlength\evensidemargin{.25in}
\setlength\oddsidemargin{.25in}


\graphicspath{{InkscapePics/}}

\newcommand{\executeiffilenewer}[3]{%
 \ifnum\pdfstrcmp{\pdffilemoddate{#1}}%
 {\pdffilemoddate{#2}}>0%
 {\immediate\write18{#3}}\fi%
}
\newcommand{\includesvg}[1]{%
 \executeiffilenewer{#1.svg}{#1.pdf}%
 {inkscape -z -D --file=#1.svg %
 --export-pdf=#1.pdf --export-latex}%
 \input{#1.pdf_tex}%
}


\DeclarePairedDelimiter\ceil{\lceil}{\rceil}


\renewcommand{\a}{\alpha}
\renewcommand{\b}{\beta}
\renewcommand{\d}{\delta}
\newcommand{\e}{\varepsilon}
\newcommand{\g}{\gamma}
\renewcommand{\k}{\kappa}
\renewcommand{\l}{\lambda}
\newcommand{\m}{\mu}
\renewcommand{\o}{\omega}
\newcommand{\p}{\rho}
\newcommand{\s}{\sigma}
\renewcommand{\t}{\tau}
\renewcommand{\th}{\theta}
\newcommand{\z}{\zeta}


\newcommand{\ba}{\bm{a}}
\newcommand{\bg}{\bm{\gamma}}
\newcommand{\bp}{\bm{p}}
\newcommand{\bq}{\bm{q}}
\newcommand{\bmm}{\bm{m}}
\newcommand{\bmu}{\bm{\mu}}
\newcommand{\bnu}{\bm{\nu}}
\newcommand{\bw}{\bm{w}}
\newcommand{\bx}{\bm{x}}
\newcommand{\bpi}{\bm{\pi}}
\newcommand{\bs}{\bm{\sigma}}
\newcommand{\bS}{\bm{\Sigma}}
\newcommand{\bth}{\bm{\theta}}
\newcommand{\bu}{\bm{u}}
\newcommand{\bv}{\bm{v}}

\newcommand{\bA}{\bm{A}}
\newcommand{\bH}{\bm{H}}
\newcommand{\bM}{\bm{M}}
\newcommand{\bW}{\bm{W}}
\newcommand{\bX}{\bm{X}}

\newcommand{\rp}{{_{r}p}}
\newcommand{\rD}{{_{r}\D}}
\newcommand{\sD}{{_{s}\D}}



\newcommand{\A}{\mathcal{A}}
\newcommand{\C}{\mathcal{C}}
\newcommand{\D}{\Delta}
\newcommand{\mD}{\mathcal{D}}
\newcommand{\E}{\mathcal{E}}
\newcommand{\F}{\mathcal{F}}
\newcommand{\G}{\mathscr{G}}
\renewcommand{\H}{\mathcal{H}}
\newcommand{\I}{\mathcal{I}}
\newcommand{\M}{\mathcal{M}}
\renewcommand{\O}{\Omega}
\newcommand{\Pt}{\mathscr{P}}
\renewcommand{\P}{\mathcal{P}}
\newcommand{\Q}{\mathcal{Q}}
\newcommand{\R}{\mathbb{R}}
\renewcommand{\S}{\mathcal{S}}
\newcommand{\T}{\mathcal{T}}
\newcommand{\Th}{\Theta}
\newcommand{\Y}{\mathcal{Y}}
\newcommand{\V}{\mathscr{V}}
\newcommand{\Z}{\mathbb{Z}}


\newcommand{\ol}{\overline}
\newcommand{\oll}[1]{\overline{\overline{ #1}}}
\newcommand{\ul}{\underline}
\newcommand{\Ex}{\mathbf{E}}
\renewcommand{\Pr}{\mathbf{P}}
\newcommand{\td}{\tilde}
\newcommand{\tr}{\triangleleft}

\newtheorem{lemma}{Lemma}
\newtheorem*{example}{Example}
\newtheorem{theorem}{Theorem}
\newtheorem{proposition}{Proposition}
\newtheorem*{definition}{Definition}
\newtheorem*{maxminrefinement}{Max-Min Threshold Refinement}
\newtheorem*{multimaxminrefinement}{Multi-Worker Max-Min Threshold Refinement}
\newtheorem*{refinement}{An Intuitive Credible Threats Refinement}
\newtheorem*{minrefinement}{Minimal Refinement}
\newtheorem{corollary}{Corollary}
\newtheorem{observation}{Observation}
\newtheorem{remark}{Remark}
\newtheorem{assumption}{Assumption}
\newtheorem*{efficiency}{Efficiency Wage Contract}

\DeclareMathOperator*{\argmax}{arg\,max}
\DeclareMathOperator*{\argmin}{arg\,min}
\DeclareMathOperator{\sgn}{sgn}



\begin{document}


\begin{center}
{\Large Econ 780 \hspace{0.5cm} HW Chapter 3}\\
\textbf{Minh Cao, Grant Smith, Ella Barnes, Alexander Erwin and Mark Coomes}\\ %You should put your name here
Due:  %You should write the date here.
\end{center}

\vspace{0.2 cm}


\subsection*{Exercises for chapter 3}


\begin{enumerate}
    \item Recall the matrix $D$ from chapter 3. Introduce the matrix $\ol{D}$: The first column is the date 0 price of the assets while the rest of the matrix is $D$. Assume as in the book $P(\o_j) > 0$ for all $j$. Consider the following market:
    \begin{align*}
    \ol{D}=
    \left[\begin{array}{llll}
    1 & 2 & 2 & 3\\
    1 & 0 & 3 & 3
    \end{array}
    \right],
    \end{align*}
    As in the book, denote the two assets by $S^1$ and $S^2$. 
      \begin{itemize}
        \item Is there a risk-free rate for this market?
        \begin{itemize}
            \item We found that the initial prices could be generated by $$(1/6-3/6 z_3, 1/3 - 3/3 z_3,z_3)$$
            We require that all three be positive, which happens when $z_3$ is between 0 and $1/3$. This means the sum of the numbers is between $1/2$ and $1/3$ and $R$ must be between 1 and 2. 
        \end{itemize}
        \item Fixing $S^1$ as numeraire, what are all the martingale measures?
        \begin{itemize}
            \item Not sure on this one.
        \end{itemize}
        \item Describe another asset $S^3$ such that the market $(S^1, S^2, S^3)$ is complete
        \emph{but not arbitrage free}.
        \begin{itemize}
            \item         An additional asset could be added to make the following matrix:
            \begin{align*}
    \ol{D}=
    \left[\begin{array}{llll}
    1 & 2 & 2 & 3\\
    1 & 0 & 3 & 3 \\
    1 & 0 & 0 & 1
    \end{array}
    \right]
    \end{align*}
        \end{itemize}

        \item Introduce the asset $X$ where $X_0 = 1$ and $X_1 = [3\ 0\ \frac{8}{3}]$ and consider the market $(S^1, S^2, X)$. Fixing $S^1$ as numeraire, is there a martingale measure? If so what is it? If not, find an arbitrage portfolio.
        \begin{itemize}
            \item The new matrix is
                        \begin{align*}
    D=
    \left[\begin{array}{lll}
    2 & 2 & 3\\
     0 & 3 & 3 \\
     3 & 0 & 8/3
    \end{array}
    \right]
    \end{align*}
    Which is non-singular. Thus, any way to get $S_0$ from linear combinations of these columns is the only linear combination that works. We calculated that linear combination, and we got negative values for the first two vectors, which means there is arbitrage. We just need to find an arbitrage portfolio. Any holding vector whose components add to zero is free because the initial prices are all 1. So we need to find a holding vector whose components add to 1 and give positive or zero dot products with the columns of $D$ (with at least one being positive). We could do that with $h = (1,-.5,-.5)$ 
        \end{itemize}
        
      \end{itemize}
      \item Consider the following market
\begin{align*}
\ol{D} = \begin{bmatrix}
1	& 2	& 2	& 2\\
2	& 8	& 4	& 2
\end{bmatrix}
\end{align*}
As in the book, denote the two assets by $S^1$ and $S^2$.

\begin{itemize}
  \item[a.] Create the normalized market by choosing $S^1$ as numeraire. Find all martingale measures. 
  
  Express you answer in the following way: Let $q_1$ denote the probability of $\o_1$ under an arbitrary martingale measure. Characterize the set of martingale measures by stating what values $q_1$ can take, and, for each such $q_1$, what are the corresponding probabilities for $\o_2$ and $\o_3$.
  
  For example, you could write: ``\textit{the set of martingale measures is} $\{(q_1, \frac{1}{2} + q_1, \frac{1}{2} - 2 q_1)\ \vert\ q_1 \in (0, \frac{1}{4})\}$." (This is the wrong answer of course).

\begin{itemize}
    \item the set of martingale measures is $\{(q_1, 1-  3q_1, 2q_1)\ \vert\ q_1 \in (0, \frac{1}{3})\}$
\end{itemize}

\begin{comment}  
  \textit{Solution}: The normalized market is
  \begin{align*}
  \ol{Z} = \begin{bmatrix}
  1	& 1	& 1	& 1\\
  2	& 4	& 2	& 1
  \end{bmatrix}
  \end{align*}
  Let $(q_1, q_2, q_3)$ be a martingale measure. The conditions on the $q$'s are: $q_i \in (0, 1)$ for all $i$, $1 = q_1 + q_2 + q_3$, and $2 = 4 q_1 + 2 q_2 + q_3$.
  
  This yields the condition $1 = 3 q_1 + q_2$ or, equivalently, $q_2 = 1 - 3 q_1$. Thus, the set of martingale measures is $\{(q_1, 1 - 3 q_1, 2 q_1)\ \vert\ q_1 \in (0, \frac{1}{3})\}$.
\end{comment}
  
  \item[b.] True of False. If we choose $S^2$ instead of $S^1$ as numeraire, we still generate the same set of martingale measures. (No need to show work.)
  \begin{itemize}
      \item True. It's just scaling the vectors, so our linear combination numbers would be different, but we scale those anyway to make the probability measure. This means the answer to this problem would be the same if we didn't scale at all or choose a numeraire.
  \end{itemize}
  
\begin{comment}
  \textit{Solution}: False. The normalized market with $S^2$ as numeraire is
  \begin{align*}
  \ol{Z} = \begin{bmatrix}
  \frac{1}{2}	& \frac{1}{4}	& \frac{1}{2}	& 1\\
  1			& 1			& 1			& 1
  \end{bmatrix}
  \end{align*}
  The probability $(\frac{1}{4}, \frac{1}{4}, \frac{1}{2})$ is a martingale measure when $S^1$ is numeraire, but not when $S^2$ is numeraire.
\end{comment}

  \item[c.] A put option $PutK$ on $S^2$ with strike price $K$ maturing at date 1 is a contingent claim with date 1 payoff $PutK_1 = \max\{K - S^2, 0\}$. What is $Put5_1(\o_i)$ for $i = 1, 2, 3$?
  \begin{itemize}
      \item $Put5_1(\o_i)=(0,1,3)$
  \end{itemize}

\begin{comment}  
  \textit{Solution}: $Put5_1 = \begin{bmatrix} 0 & 1 & 3 \end{bmatrix}$.
\end{comment}
  
  \item[d.] Suppose the date 0 price of $Put5$ is $Put5_0 = \frac{7}{8}$. Is the market $(S^1, S^2, Put5)$ arbitrage free?
  \begin{itemize}
      \item i THINK THERE IS SOMETHING WRONG WITH THIS ANSWER!!! 
      \item No it is not. To prove this, we try to create $7/8$ with the dot product of our martingale measures above and $(0,1,3)$.  This gives us $7/8 = 1+3q_1$, but $q_1$ must be positive, so we have a contradiction.
  \end{itemize}

\begin{comment}  
  \textit{Solution}: To show $(S^1, S^2, Put5)$ is arbitrage free amounts to finding a martingale measure $Q$ of the normalized market $(1, S^2/S^1)$ with $S^1$ as numeraire such that, in addition, $\frac{7}{8} = \frac{1}{2} \Ex^Q Put5_1$.
  
  The above equation plus the characterization of martingale measures for the normalized market $(1, S^2/S^1)$ from part (a) implies $\frac{7}{4} = 1 \cdot (1 - 3q_1) + 3 \cdot (2 q_1)$, which yields $q_1 = \frac{1}{4}$ and a probability $(\frac{1}{4}, \frac{1}{4}, \frac{1}{2})$. Thus, $(S^1, S^2, Put5)$ is arbitrage free because $(\frac{1}{4}, \frac{1}{4}, \frac{1}{2})$ is the unique martingale measure of the corresponding normalized market $(1, S^2/S^1, Put5/S^1)$ with $S^1$ as numeraire.
\end{comment}
  
  
  \item[e.] Continuing to assume the date 0 price of $Put5$ is $\frac{7}{8}$. What is the arbitrage free date 0 price of $Put4$?
  \begin{itemize}
      \item The replicating holding vector is $(-4/3,1/3,4/3)$, and $hS_0$ is 0.5.
  \end{itemize}
 
\begin{comment} 
  \textit{Solution}: The arbitrage free date 0 price of $Put4$ is $\frac{1}{2} \Ex^{(\frac{1}{4}, \frac{1}{4}, \frac{1}{2})} Put4_1 = \frac{1}{2} (\frac{1}{2} \cdot 2) = \frac{1}{2}$.
\end{comment}
\end{itemize}
  \item Exercise 3.2.
  \begin{itemize}
      \item If $X$ is replicable, then:
      $$\exists h; hS_1 = X$$
      \item Now, using the definition of the value of a portfolio:
      $$V_0^h = hS_0$$
      \item If there is no arbitrage, then there is a column vector, $z$ such that $S_0$ can be formed by $z$'s scaling of $S_1$'s columns. Thus,
      $$\exists z; S_0 = S_1z$$
      \item We can use this equation with the second bullet to get:
      $$V_0^h = hS_0 = hS_1z$$
      \item If we want to turn z into a vector that scales to 1 (to make it a probability measure), we would want to divide its components by the sum of all the components, i.e. to normalize it. So to do this, we can multiply and divide by the sum, and we can call the new normalized z, q:
      $$V_0^h = hS_0 = hS_1z \frac{\Sigma z_i}{\Sigma z_i} = \Sigma z_i * hS_1q $$
      \item But we know that $\Sigma z_i$ has a name, and it is $\beta$, which is also $1/1+R$. Thus,
      $$V_0^h = hS_0 = hS_1z \frac{\Sigma z_i}{\Sigma z_i} = \Sigma z_i * hS_1q =  \frac{1}{1+R}hS_1q$$
      \item Now we use our existence in the very first step, which is our replicating portfolio.  Thus, we get (and removing the intermediate steps):
      $$V_0^h =  \frac{1}{1+R}hS_1q = \frac{1}{1+R}Xq $$
      \item And lastly, we say that $Xq$ is the expectation of $X$ under $S$:
      $$V_0^h = \frac{1}{1+R}Xq = \frac{1}{1+R}\mathbb{E}^Q[X]$$
  \end{itemize}
  \item Exercise 3.3.
  \begin{itemize}
      \item We want to prove that the arbitrage free price of an claim $X$ is given by:
      $$\Pi (0,X) = \mathbb{E}^P[\Lambda (\omega) X(\omega)]$$
      \item We start with the definition of $\Lambda (\omega)$:
      $$\Lambda (\omega) = \frac{1}{1+R} L(\omega) = \frac{1}{1+R} \frac{q_i}{p_i}$$
      \item We now substitute this into the first equation:
            $$\Pi (0,X) = \mathbb{E}^P[\frac{1}{1+R} \frac{q_i}{p_i} X(\omega)] = \frac{1}{1+R}\mathbb{E}^P[ \frac{q_i}{p_i} X(\omega)]$$
        \item We now expand the expectation:
        $$\Pi (0,X)  = \mathbb{E}^P[\Lambda (\omega) X(\omega)] = \frac{1}{1+R}\mathbb{E}^Q [X] $$
        Which was our goal.
  \end{itemize}

\end{enumerate}


\end{document}